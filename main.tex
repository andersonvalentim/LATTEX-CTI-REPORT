\documentclass[12pt]{article}
\usepackage[utf8]{inputenc}
\usepackage[brazil]{babel}
\usepackage{geometry}
\geometry{a4paper, margin=2.5cm}
\usepackage{titlesec}
\usepackage{hyperref}
\usepackage{xcolor}
\usepackage{graphicx}
\usepackage{fancyhdr}
\usepackage{enumitem}

\titleformat{\section}{\large\bfseries}{\thesection.}{1em}{}

\pagestyle{fancy}
\fancyhf{}
\rhead{Relatório de CTI}
\lhead{ACV Tecnologia}
\rfoot{\thepage}
\cfoot{TLP: AMBER \quad |\quad Página \thepage}
% Altere aqui o nível do TLP conforme necessário: WHITE / GREEN / AMBER / RED
% --------------------------------------------------------
\title{\textbf{Modelo de Relatório de Cyber Threat Intelligence (CTI)}}
\author{}
\date{}

\begin{document}

\maketitle
\hrule
\vspace{0.5cm}

\section{Identificação do Relatório}
\begin{itemize}[leftmargin=*]
    \item \textbf{ID do Relatório:} CTI-AAAA-MM-\#\#\#
    \item \textbf{Classificação da Informação:} [Público / Restrito / Confidencial / TLP:WHITE/AMBER/RED]
    \item \textbf{Data do Relatório:} DD/MM/AAAA
    \item \textbf{Autor(es):} Nome(s) / Equipe
    \item \textbf{Contato:} Email ou canal seguro
\end{itemize}

\section{Resumo Executivo}
Breve descrição do incidente, campanha ou ameaça analisada, em linguagem acessível a executivos e stakeholders não técnicos.

\textit{Exemplo:} Identificamos uma campanha de phishing visando setores de agro negócio do Brasil, associada ao grupo TA505.

\section{Descrição Técnica}
\begin{itemize}[leftmargin=*]
    \item \textbf{Ameaça Observada:} [Malware, Phishing, APT, Ransomware etc.]
    \item \textbf{Data/Hora da Detecção:} 
    \item \textbf{Vetores de Ataque:} 
    \item \textbf{Alvos Potenciais/Afetados:} 
    \item \textbf{Indicadores de Comprometimento (IOCs):}
    \begin{itemize}
        \item IPs:
        \item Domínios:
        \item Hashes:
        \item URLs:
        \item Regras YARA/Sigma:
    \end{itemize}
    \item \textbf{Técnicas, Táticas e Procedimentos (TTPs):} (MITRE ATT\&CK)
\end{itemize}

\section{Análise de Atribuição (se aplicável)}
\begin{itemize}[leftmargin=*]
    \item \textbf{Possível Grupo Ator:} 
    \item \textbf{Motivação Provável:} [Financeira / Espionagem / Hacktivismo / etc.]
    \item \textbf{Evidências da Atribuição:} 
\end{itemize}

\section{Avaliação de Impacto}
\begin{itemize}[leftmargin=*]
    \item \textbf{Impacto Potencial:} 
    \item \textbf{Probabilidade de Ocorrência:} [Alta / Média / Baixa]
    \item \textbf{Relevância para o Negócio:} 
\end{itemize}

\section{Recomendações}
\begin{itemize}[leftmargin=*]
    \item \textbf{Ações de Contenção e Mitigação:} 
    \item \textbf{Correções Necessárias:} 
    \item \textbf{Boas Práticas e Lições Aprendidas:} 
    \item \textbf{Sugestão de Bloqueios:} (IP, domínio, etc.)
\end{itemize}

\section{Anexos}
\begin{itemize}[leftmargin=*]
    \item Logs
    \item Capturas de Tela
    \item Fluxogramas
    \item Arquivos PCAP
    \item Referências externas (CVE, links OSINT, etc.)
\end{itemize}

\section{TLP e Política de Compartilhamento}
\begin{itemize}[leftmargin=*]
    \item \textbf{Nível TLP Aplicado:} 
    \item \textbf{Permissões de Redistribuição:} 
    \item \textbf{Orientações para Compartilhamento Controlado:} 
\end{itemize}

\end{document}
